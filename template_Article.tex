\documentclass[12pt]{article}
\usepackage[a4paper, left=30mm, top=10mm, bottom=10mm, right=20mm]{geometry}
\usepackage[T1]{fontenc}
\usepackage[scaled]{helvet}
%\usepackage{fontspec}
%\setmainfont{uarial}

% Load blindtext package for dummy text
\usepackage{blindtext}

% Load the parskip package with skip and indent options
\usepackage[skip=8pt plus1pt, indent=40pt]{parskip}
\usepackage{indentfirst}
% Load the setspace package
\usepackage{setspace}

\usepackage{anyfontsize}




% Customs
\renewenvironment{abstract}
{\small
	\begin{center}
		\bfseries \abstractname\vspace{-.5em}\vspace{0pt}
	\end{center}
	\list{}{
		\setlength{\leftmargin}{.0cm}%
		\setlength{\rightmargin}{\leftmargin}%
	}%
	\item\relax}
{\endlist}
%End Customs

%opening
\title{Math for Machine Learning}

\author{Chico Freitas\footnote{Professor de Física no IFPI-\textit{Campus} Teresina Central, prof.ofreitas@gmail.com}}

\begin{document}

%phv: Helvética
{\fontfamily{phv}\selectfont
	\begin{spacing}{1.5}
	\maketitle
	
	\begin{abstract}
		\begin{singlespace}
			\noindent
			%\begin{large}
		
				Este documento apresenta o modelo sugestivo de formatação para artigos
			científicos da revista científica - Faculdade CET. O resumo é elemento obrigatório
			constituído de uma sequência de frases objetivas e não uma enumeração de
			tópicos, no mesmo idioma do trabalho, não se deve ultrapassar a 250 palavras,
			sintetizando o tema em questão, objetivo do estudo, a metodologia e as
			considerações finais a que se chegou. Deve-se evitar frases longas e não se recorre
			a citações ou uso de qualquer tipo de ilustração (gráfico, tabela, fórmulas). Esse
			resumo deve ficar na primeira página em Fonte Arial 12, espaçamento simples (1,0).
			Para as palavras-chave recomendamos um parágrafo único com 3 (três) a 5 (cinco)
			palavras separadas por ponto-e-vírgula, com a primeira letra de cada palavra em
			maiúsculo e finalizadas por ponto, conforme o texto.
			%\end{large}
		\end{singlespace}

		\textbf{Keywords:} Math, Machine Learning, Linear Algebra, Probability, Programming.
	
	\end{abstract}
	
	\section*{ \fontfamily{phv}\selectfont Introduction}
			Este documento está escrito de acordo com o modelo indicado para o artigo,
			assim, serve de referência, ao mesmo tempo em que comenta os diversos aspectos
			da formatação. Observe as instruções e formate seu artigo de acordo com este
			padrão. A redação do artigo deve considerar o público ao qual se destina. A2
			linguagem
			será
			gramaticalmente
			correta,
			precisa,
			coesa,
			coerente
			e,
			preferencialmente, em terceira pessoa ou utilizando a impessoalização textual\par
			
			\blindtext[1]\par
			
			\blindtext[1]
		
	\section{Methods}
			\blindtext[1]
	
	\section{Development}
			
			\blindtext[1]
			
			\begin{quotation}
				
				\begin{singlespace}
					\noindent
					\blindtext[1]
					(DOE, 2022, p. 20)
				\end{singlespace}
			
			\end{quotation}
	
	\section{Results}
		
			\blindtext[1]\par
			
			\blindtext[1]\par
			
			\blindtext[1]
	
	\section{Conclusions}

			\blindtext[1]\par
			
			\blindtext[1]\par
			
			\blindtext[1]
	\end{spacing}
}% end font encode

\end{document}
